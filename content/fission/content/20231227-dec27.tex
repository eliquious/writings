
\addcontentsline{toc}{section}{December 27, 2023}
\section*{December 27, 2023}

% \subsection*{Confession \& Reflection}

\subsection*{Thanksgiving}
\textit{Yesterday, I was thankful to God for... This was a blessing because...} \\

\noindent Yesterday, I was thankful for the opportunity to eat at a mediteranean restaurant in Canton. This is a blessing because I can afford to travel for lunch and to get out of the house. It also brings a fondness for the Middle East.

\subsection*{God's Pleasure}
\textit{Yesterday, Jesus was pleased with me because...} \\

\noindent I think God was pleased with me yesterday with my conversation with Kevin Kirchner. We talked about the Maranatha Library and caught up about what the Lord's doing. I think the Lord was pleased with our conversation.

\subsection*{Disobedience}

\textit{Yesterday, I was disobedient to God when...} \\

\noindent Yesterday, I was disobedient to God when I stumbled.

I've also been avoiding the conversation with the unbeliever on Twitter because of the emotions involded. I don't want him to disrupt my peace. I don't think it's necessarily disobedience but I've been avoiding him. \\

\subsection*{Reflection}

\textit{How I could have done better?} \\

I need to remember to make no provision for the flesh (Romans 13:14).

\subsection*{Repentance / Reconciliation}

\textit{What can I do to reconcile / ask forgiveness?} \\

I have repented before the Lord for stumbling. I could also avoid conversations with such unbelievers whose hearts are hardened.

% ==============================================================
% Scripture quotation
\subsection*{Scripture Reading - 1 Timothy 2}

\vspace*{0.5cm}
\begin{center}
	\normalsize{\parbox{10.5cm}{
		\begin{raggedright}
		{\normalsize 
			\textit{First of all, then, I urge that supplications, prayers, intercessions,
            and thanksgivings be made for all people, for kings and all who are in 
            high positions, that we may lead a peaceful and quiet life, godly and 
            dignified in every way. This is good, and it is pleasing in the sight of 
            God our Savior, who desires all people to be saved and to come to the 
            knowledge of the truth.}
		}

		\vspace{.25cm}\hfill{---1 Timothy 2:1-4}
		\end{raggedright}
	}
}
\end{center}

Paul seems to indicate here that the ability to lead a peaceful and quite life is dependent upon the prayers of the saints for those in power. And that these prayers should be driven by our desire to see them saved and come to the knowledge of the truth because that is God's desire.

\vspace*{0.5cm}
\begin{center}
	\normalsize{\parbox{10.5cm}{
		\begin{raggedright}
		{\normalsize 
			\textit{For there is one God, and there is one mediator 
            between God and men, the man Christ Jesus, who gave himself as a ransom 
            for all, which is the testimony given at the proper time. For this I was 
            appointed a preacher and an apostle (I am telling the truth, I am not lying), 
            a teacher of the Gentiles in faith and truth.}
		}

		\vspace{.25cm}\hfill{---1 Timothy 2:5-7}
		\end{raggedright}
	}
}
\end{center}

Paul is distinctly separating the roles of preacher, apostle and teacher in these verses. They are not the same position within the body and Paul was gifted to be operating in three different roles or offices within the body of Christ.

% ==============================================================
